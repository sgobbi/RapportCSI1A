\section{Présence et incarnation en VR}


\subsection{Sentiment d'incarnation dans un avatar}
Le sentiment d’embodiment dans un corps virtuel correspond à l’impression d’habiter, de posséder et de 
contrôler un corps numérique comme s’il s’agissait de son propre corps réel. Ce phénomène repose sur 
trois composantes principales : l’auto-localisation (le fait de ressentir que l’on se trouve à l’endroit 
occupé par le corps virtuel), le sentiment de possession du corps (body ownership), et le sentiment d’agence
 (agency), c’est-à-dire le ressenti que l’on contrôle les mouvements de ce corps. Ces composantes sont 
 activées par des signaux sensoriels synchrones et cohérents, notamment visuels, proprioceptifs et moteurs.
  Par exemple, lorsque les mouvements réels de l’utilisateur sont fidèlement reproduits dans l’avatar à l’écran,
le cerveau intègre ces signaux comme s’ils provenaient du corps réel, générant ainsi l’illusion d’habiter
ce corps virtuel. Cette illusion peut être si forte qu’elle entraîne des réponses psychophysiologiques 
cohérentes avec ce que vivrait le corps réel dans la même situation (par exemple, une réaction de peur
    face à une menace dirigée vers le corps virtuel). Des recherches, notamment celles de Kilteni et Slater \cite{Kilteni2012Embodiment}, 
    ont montré que ce sentiment peut être renforcé par des interactions multisensorielles (comme le retour
tactile ou audio), par la cohérence morphologique entre le corps virtuel et le corps réel, mais aussi 
      par des facteurs contextuels, comme le scénario ou l'environnement dans lequel l'expérience se déroule.
       L’embodiment dans un avatar ne se limite donc pas à une simple illusion visuelle : il engage profondément 
       les représentations corporelles internes de l’utilisateur, au point d’influencer ses comportements, 
       ses émotions, et même son image de soi.


\subsection{Sentiment de présence dans un environnement virtuel}
Notre travail se focalisant sur les effets de l'environnement sur le jeu du comédien, nous nous penchons sur le 
sentiment de présence dans un environnement virtuel, qui est définie comme l'expérience subjective d'être dans un lieu ou un environnement, 
même si l’on se trouve physiquement dans un autre, \cite{WitmerSinger1998PresenceQuestionnaire}.
Ce sentiment est défini non pas comme une illusion cognitive mais perceptuelle par 
Slater et ses collègues, \cite{Slater2018Immersion}. En effet, lorsqu'un immersant est immergé dans un monde virtuel
par n'importe quel dispositif immersif, il est conscient que cet espace visionné et le scénario qui s'y 
déroule ne sont pas réalité. C'est alors 
que se produit consciement la suspension de l'incrédulité qui lui permet de croire à cette illusion perceptuelle
et d'avoir des réactions naturelles à ce qu'il se passe autour de lui. C'est ce qu'on 
appelle la "place illusion" et la "plausibility illusion", \cite{Slater2009PlaceIllusion}. \\
Certains facteurs vont venir influencer ce sentiment de présence et peuvent être structurés d'une façon similaire à celle
de l'incarnation dans un corps virtuel. Les deux éléments marquants sont la sensation de localisation --- avoir l’impression que l’on
se trouve physiquement dans l’environnement que l’on visualise --- et celle d’agency --- sensation de contrôle
sur cet environnement. En effet, si notre presence, même virtuelle, a une influence visible sur l’environnement qui nous entoure,
on s’y sentira forcément plus présent. 
Finalement, cette sensation de présence a certains effets sur l'immersant, nous permettant donc de la mesurer ou du 
moins de la constater chez les participants. Ces effets se manifestent à la fois de manière immédiate et à long terme : on parle 
de sentiment de présence à partir du moment où, du point de vue de l’utilisateur, l’environnement virtuel prend le pas sur l’environnement réel
 (par exemple, en termes de réaction aux stimuli). Ce sentiment se prolonge également après l’expérience, à travers l’impression
  d’avoir réellement visité un lieu et vécu un scénario, et non simplement de l’avoir observé. 


  \vspace{\baselineskip}

Ce sont donc sur des facteurs portant à la fois sur le sentiment de présence et le sentiment d'incarnation dans 
un corps virtuels que nous allons devoir nous pencher afin d'améliorer le jeu des acteurs dans cet espace et leur acceptation
de cet outil de travail. 
