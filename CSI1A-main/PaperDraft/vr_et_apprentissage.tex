\section{Réalité virtuelle et apprentissage de geste}

Nous avons abordé pour le moment avant tout la partie de pédagogie théorique, autour de l'ITS et de l'apprentissage implicite pour la notion de geste expert. 
Un autre mot de fort de notre sujet est la réalité mixte, car cela sera l'outil retenu pour notre expérimentation.
Nous allons commencer par étudier ce qui se fait déjà dans l'apprentissage de geste en réalité virtuelle, avant faire un petit tour d'ensemble de l'utilisation des différentes technologies de réalité mixte.
Nous attarderons ensuite sur quelques méthodes qui ont retenu notre attention. 

\subsection{Généralités de l'apprentissage de geste en réalité virtuelle}

On peut retrouver de nombreux exemples d'apprentissage de gestes dans la littérature ou l'industrie, que ce soit pour de l'entrainement sportif ou dans un contexte industriel.
Cependant, ces gestes n'impliquent en général pas de \textit{négociation avec la matière}, comme on peut le retrouver dans l'artisanat : 
les gestes aujourd'hui appris en réalité virtuelle sont des postures d'arts martiaux, des interactions avec des machines ou consignes de sécurité principalement. 
On retrouve également quelques applications dans des sports de raquette (tennis, tennis de table) ou en médecine. \\

Dans la plupart des études, on retrouve des métriques évaluant la précision obtenue par l'apprentissage en réalité virtuelle, mais plus le transfert et la rétention sont plus rarement étudiées \cite{ComplexSkillVRLearningReview.Levac+}.
Cependant, dans le cadre d'un apprentissage de type formation, le transfert et la rétention sont à nos yeux les plus intéressantes : l'apprenant pourra-t-il faire fructifier son apprentissage en réalité virtuelle dans son apprentissage global de son métier d'artisan. 
D'après \cite{TrainingVR.Pastel+}, l'apprentissage en réalité virtuelle aurait les mêmes propriétés de transfert que la vidéo pour la réalisation d'un mouvement sportif dans des conditions d'utilisation similaire.  \\

Une autre force de la réalité virtuelle est d'offrir un apprentissage actif, qui implique l'apprenant et lui permet d'être acteur de son apprentissage. 
Cela permet d'augmenter l'autonomie de l'apprentissage, la confiance en soi, et d'avoir ensuite un apprentissage plus efficace, au moins sur la précision \cite{ApplicationVRForTraining.Tang}. 
On retombe d'ailleurs sur certains des avantages mises en avant dans la théorie OPTIMAL pour l'apprentissage implicite \cite{OPTIMAL.Wulf+}. \\

Les environnements virtuels sont aussi appréciés dans l'entrainement moteur car ils permettent d'avoir un contrôle total plus précis de l'environnement que cela est possible dans la réalité \cite{ComplexSkillVRLearningReview.Levac+}.
Cela peut permettre de s'assurer d'avoir l'environnement le plus adéquat pour l'apprentissage sans aléas qui peuvent prendre place dans la réalité, ou alors au contraire de simuler des situations rares \cite{InvestigateAcquireSkill.Mangalam+}.
Ainsi peut-on s'entrainer dans des situations dangereuses ou inhabituelles pour être prêt à y faire face le jour où celles-ci se présenteront, alors qu'il serait plus difficile d'imaginer s'y entraîner de façon pratique dans la réalité. 
Enfin, on peut voir l'environnement contrôlé comme une opportunité de soumettre l'apprenant à des lois physiques différentes pour travailler sa réaction ou des pans précis du geste \cite{ComplexSkillVRLearningReview.Levac+}. \\

\subsection{Réalité mixte et apprentissage}

Nous avons pour le moment parlé tout le temps de réalité virtuelle, mais il convient d'aussi discuter des autres modalités de la réalité étendue, quand bien même elles sont aujourd'hui moins répandues que la réalité virtuelle. 
La réalité virtuelle offre l'avantage de couper l'apprenant du monde extérieur, ce qui permet une plus grande concentration en général. 
On peut cependant quand même observer une peur des obstacles dans l'environnement réel, qui se retrouvent masqués en réalité virtuelle \cite{EnhancingTaiChi.Tian+}. \\

La réalité mixte permet une meilleure conscience de son environnement, et offre une composante sociale, en particulier en cas d'apprentissage à plusieurs, leur permettant d'interagir plus facilement et naturellement \cite{EnhancingTaiChi.Tian+}.
Il n'y a cependant pas d'avantage direct sur les performances par rapport à la réalité virtuelle dans les articles que nous avons lus. \\

Enfin, nous n'avons pas encore lu d'articles portant sur l'efficacité de la réalité augmentée dans l'apprentissage du geste. 

\subsection{Visualisations spéciales du mouvement pour l'apprentissage en VR}
