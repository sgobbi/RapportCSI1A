\subsubsection{Immersion VR et sentiment de présence}
Une des premières conclusions que nous voulions tirer de cette experience relève du sentiment de présence
des comédiens dans les environnements dans lequel ils évoluent et répètent leurs monologues, ainsi que dans
le role qu'ils incarnent. Pour ce faire, nous avons utilisé des métriques connues de la présence en réalité virtuelle à travers
le questionnaire de présence IGroup Presence Questionnaire; \cite{Schubert}, dont on a extrait
certaines questions. 
L'analyse des réponses à ces questionnaires une bonne acceptation des environnements proposés. Les éléments
visuels ont été perçus comme immersifs, donnant l’impression de se trouver dans un autre monde, propice à 
l’exploration et à l’évasion du monde réel. 
\\
En plus de ces questionnaires, certains éléments nous permettent de dire que les comédiens avaient un sens 
élevé de présence dans ces environnements. En effet, les espaces traversés lors de l'immersion ont laissé une forte
empreinte spatiale preuve d'un grand sentiment de présence dans l'environnement virtuel, \cite{Khosravi}, de telle sorte que les participants ont pu décrire avec précision l'environnement dans lequel 
ils étaient, en pointant du doigt et en se tournant vers les éléments évoqués, comme s'ils étaient encore présents autour
d'eux, bien après la fin de l'expérience.  De plus, dans le cas de l'environnement abstrait, la foule a été ressentie comme
une menace qui a générée de l'angoisse subjective chez certains participants, ce qui montre un grand sentiment de présence
et d'acceptation physique du monde virtuel \cite{Zhang}. Leurs intéractions avce les avatars lorsqu'il y en avait témoignent elles aussi
d'un fort sentiment de présence et d'acceptation du monde virtuel puisqu'elles respectaient les normes sociales de proxémie 
habituellement observées dans le monde réel, \cite{Wilcox}. Par exemple, lorsqu'ils e téléportaient trop près de ceux-ci, 
ils s’écartaient rapidement en s’excusant. 
\\
Cependant, plusieurs éléments de l’expérience ont été identifiés comme ayant réduit le sentiment de présence chez les participants. 
Un point récurrent concerne l'absence significative de lien de causalité entre les actions des utilisateurs et leur environnement virtuel. 
En particulier, lors des immersions dans des environnements imposés, les participants ont noté que leurs gestes ou prises d’initiatives n’avaient 
aucun impact perceptible sur leur environnement, ce qui a contribué à une sensation de passivité et de détachement, voir même d'invisibilité. 

En effet, dans la scène imposée du métro, un personnage animé est assis sur un banc. Cependant, il devient rapidement évident que ses mouvements sont 
préprogrammés et qu’il ne réagit en aucune manière au jeu de l’acteur. Cette absence de réciprocité a été soulignée par Matisse, qui a exprimé des 
difficultés à jouer face à un partenaire virtuel dépourvu de toute capacité d’interaction.
De même, dans la scène plus abstraite, Matisse a exprimé un désir de pouvoir interagir physiquement avec les éléments de l’environnement, 
notamment le brouillard. Elle souhaitait que ses mouvements puissent produire une forme de réponse, que ce soit visuelle ou sonore, ce qui aurait renforcé 
l’illusion d’être véritablement présente dans cet espace.
Finalement, lors de la phase plus libre de l’atelier, où les participants pouvaient choisir eux-mêmes les environnements dans lesquels ils souhaitaient s’immerger, 
Jules a exprimé le souhait d’ajouter de nombreux éléments et événements sur lesquels il pouvait exercer un certain contrôle (des statues qui apparaissent doucement quand
il en parle puis qui explosent quand il les touche). 

\subsubsection{Construction et modification de son environnement}
Dans la deuxième partie de l'étude, nous avons voulu tester la pertinence de laisser la liberté aux acteurs de construire leurs
propres environnements virutels, et de voir quelles conséquences cette liberté peut avoir sur leur quête de sens du texte, leur interprétation,
leur jeu, leur acceptation de l'environnement d'immersion, et leur sentiment de présence dans celui-ci. Nous en avons conclu que 
dans le cadre d’un travail de répétition en réalité virtuelle, permettre aux comédiens de construire, modifier et moduler leur environnement d’immersion
s’avère être un levier essentiel. Cette liberté de création prend tout son sens dans le contexte de l’étude d’un texte théâtral : l’espace scénique 
virtuel devient alors un terrain d’expérimentation qui évolue en fonction de la compréhension progressive du texte, des interprétations qui en 
émergent, et de l’imaginaire propre du comédien — ou du metteur en scène, selon qui initie les décisions.
\\ 
Les principales questions auxquelles on a voulu répondre sont les suivantes:
\begin{itemize}
    \item Quel est le bon équilibre entre liberté et guidage?
    \\Il nous apparait que la liberté totale n'est pas forcément toujours source de création et n'entraine pas 
    automatiquement une profusion de création. En effet, les comédiens ont considérés que la première étape de l'étude 
    avait été très importante et formatrice. Le passage initial lors duquel les environnements étaient entièrement imposés, bien que passif, 
    s’est révélé précieux : il a permis aux comédiens de se familiariser avec les potentialités offertes par la réalité virtuelle et de se 
    constituer un référentiel. Le catalogue d’environnements et d’objets mis à disposition a joué un rôle similaire : en exposant les 
    utilisateurs à des options auxquelles ils n’auraient peut-être pas pensé spontanément, il a nourri leur imaginaire. Jules, par exemple, 
    a pu utiliser des statues issues du catalogue pour incarner à la fois la Sorbonne — comme symbole institutionnel — et les héros que son 
    personnage devait rejeter. De plus, lors du troisième jour, laissé largement à l’initiative des participants, une seule demande de nouvel 
    élément non présent dans le catalogue a été formulée (le « métro hippopotame »). Cela suggère que la créativité en réalité virtuelle 
    peut bénéficier d’un cadre souple mais structurant, combinant liberté de choix et incitations extérieures non intrusives. Les comédiens 
    ont d’ailleurs exprimé une ouverture à recevoir des suggestions d’un agent tiers (comme un metteur en scène ou une intelligence artificielle), 
    à condition que ces propositions restent discrètes et respectueuses de leur autonomie, à l’image de directions de jeu murmurées.
    \item Quelles sont les modalités et les interfaces de création adéquates? 
    \\Il existe deux éléments qui ont un impact important sur la création de 
    l'environ- \\ nement: le contenu lui-même (les éléments de décor sur 
    lesquelles l'utilisateur a du pouvoir) et l'interface (la façon dont l'utilisateur va pouvoir déclencher ces modifications et interagir avec 
    son environnement). 
    \\En termes de contenu, les participants ont montré un besoin marqué de pouvoir ajuster à la fois l’ambiance générale (lumière, temporalité) 
    et des éléments précis. Matisse, par exemple, a jugé l’amphithéâtre trop vaste pour la scène introspective qu’elle devait interpréter : elle 
    avait besoin de se recentrer physiquement avant de pouvoir s’immerger dans le jeu. De son côté, Jules a exprimé la nécessité de plonger son 
    environnement dans la nuit ou de baisser les lumières pour déclencher un état émotionnel propice à l’incarnation. La possibilité d'ajouter
    des objets précis a aussi été jugée importante puisqu'elle permet de construire son espace en suivant un cheminement struturé: les comédiens commencent 
    par disposer les éléments qui structurent l’espace, puis ceux qui orientent le jeu (notamment en termes d’adresse), avant d’ajouter les éléments symboliques tirés de l'interprétation personnelle du texte. 
    Par exemple, dans une des experiences, Matisse a ajouté une multitude de caméras de surveillance, ce qui lui a permis à la fois d'orienter son jeu vers celles-ci, et 
    de travailler l'idée de surveillance présente tout au long du texte. Ils ont exprimé l'envie d'images encore plus puissantes sur le plan sensoriel qui viendraient renforcer l’impact dramaturgique de certaines scènes, notamment Jules 
    qui a eu l'idée de sang qui dégoulinerait des murs.
    \\En plus de ce qu'on leur avait proposé dans la première partie de l'étude, les comédiens ont formulé le désir de pouvoir déclencher des événements, 
    qu’il s’agisse d’animations simples ou de transitions majeures. Ces mouvements créent du rythme et permettent de structurer le texte, de marquer ses ruptures, ses temps forts. 
    Jules notamment a demandé a ce que les lumières baissent en luminosité lentement afin de créer une transition vers son jeu, puis une animation de statue qui doucement peuplent tout l'espace 
    autour de lui et finalement l'arrivée lente du train à la fin du monologue. 
    \item Quels effets cette création a-t-elle eu sur l'immersion et le sentiment de présence?
    \\ La construction de leur propre environnement a significativement augmenté le sentiment de présence et l’immersion des participants. Contrairement aux situations plus passives, 
    les moments où les comédiens étaient en action, en train de créer ou de modifier leur espace, ont engendré une plus grande perte de repères avec le monde extérieur — un indicateur 
    fort d’immersion.
    Sur le plan dramaturgique, cette phase de création a aussi permis une incarnation et une compréhension plus profonde du role puisque chaque choix spatial devenait porteur de sens:
    il interrogeait l’interprétation du texte et favorisait l’émergence d’un jeu personnel. Jules, par exemple, expliquait que chaque décision — chaque objet ou ambiance choisie — impliquait 
    une question sous-jacente : Pourquoi ce choix? Que dit-il du personnage ou de la situation? Matisse, quant à elle, soulignait l’importance de pouvoir insérer des éléments qui la « faisaient jouer », 
    c’est-à-dire qui stimulaient son imaginaire, son corps, et sa capacité à entrer dans la situation dramatique.   
\end{itemize}

\subsubsection{Durabilité des effets de l'immersion}
Enfin, nous avons souhaité observer la durabilité des effets de l’immersion, tant d’une séance à l’autre que dans le passage du jeu en réalité virtuelle au jeu au plateau. Il est apparu que les immersions 
successives ne s’inscrivaient pas comme des expériences isolées, mais comme des fragments d’un même parcours sensoriel et imaginaire. Jules, par exemple, a abordé sa seconde immersion dans l’amphithéâtre 
comme une sorte de rêve issu de la station de métro où il se trouvait juste avant. Plutôt que de vivre cette transition comme une rupture, il l’a intégrée de manière fluide à son jeu, en considérant que son personnage 
se laissait envahir par son imaginaire tout en restant physiquement ancré dans la station de métro. Ce chevauchement des strates d’immersion témoigne d’un prolongement actif de l’expérience précédente dans la suivante.

Sur le plateau, après les immersions, plusieurs éléments sont revenus de façon immédiate, sans que les comédiens aient besoin de les convoquer consciemment. Les images perçues dans le casque se sont réactivées spontanément, 
comme si elles s’étaient imprimées dans le corps et dans le jeu. Au-delà du souvenir visuel, ce sont également des sensations corporelles plus subtiles qui ont ressurgi : ainsi, la sensation de la foule qui traverse et transperce,
 vécue en immersion, a pu être retrouvée lors du travail physique au plateau. Ces réminiscences suggèrent que l'immersion en réalité virtuelle ne se contente pas de stimuler l'imaginaire à court terme, mais qu'elle laisse une 
 empreinte durable, à la fois sensorielle et dramaturgique, dans le processus de création du comédien.
 \\

