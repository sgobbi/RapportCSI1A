\subsubsection{Immersion VR et sentiment de présence}
Une des premières conclusions que nous voulions tirer de cette experience relève du sentiment de présence
des comédiens dans les environnements dans lequel ils évoluent et répètent leurs monologues, ainsi que dans
le role qu'ils incarnent. Pour ce faire, nous avons utilisé des métriques connues de la présence en réalité virtuelle à travers
le questionnaire de présence IGroup Presence Questionnaire; \cite{Schubert}, dont on a extrait
certaines questions. 
L'analyse des réponses à ces questionnaires une bonne acceptation des environnements proposés. Les éléments
visuels ont été perçus comme immersifs, donnant l’impression de se trouver dans un autre monde, propice à 
l’exploration et à l’évasion du monde réel. 
\\
En plus de ces questionnaires, certains éléments nous permettent de dire que les comédiens avaient un sens 
élevé de présence dans ces environnements. En effet, les espaces traversés lors de l'immersion ont laissé une forte
empreinte spatiale, de telle sorte que les participants ont pu décrire avec précision l'environnement dans lequel 
ils étaient, en pointant du doigt et en se tournant vers les éléments évoqués, comme s'ils étaient encore présents autour
d'eux, bien après la fin de l'expérience. De plus, dans le cas de l'environnement abstrait, la foule a été ressentie comme
une menace qui a générée de l'angoisse subjective chez certains participants, ce qui montre un grand sentiment de présence
et d'acceptation physique du monde virtuel \cite{Zhang}. Leurs intéractions avce les avatars lorsqu'il y en avait témoignent elles aussi
d'un fort sentiment de présence et d'acceptation du monde virtuel puisqu'elles respectaient les normes sociales de proxémie 
habituellement observées dans le monde réel, \cite{Wilcox}. Par exemple, lorsqu'ils e téléportaient trop près de ceux-ci, ils s’écartaient rapidement en s’excusant. 


\subsubsection{Construction et modification de l'environnement d'immersion}
\subsubsection{Durabilité des effets de l'immersion}



