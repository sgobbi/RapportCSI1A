L'étude pilote s'est déroulée en 2 temps distincts. Le premier temps suivait un protocole clair tandis que le deuxième prenait la forme d'un atelier de découverte et de co-création
lors duquel on travaillait main dans la main avec les acteurs afin de répondre à leurs envies et leurs besoins particuliers. 

Le déroulé de la première étape fut le suivant: 

\begin{itemize}
    \item Phase préalable: les comédiens reçoivent le monologue à travailler chez eux afin d'apprendre le texte uniquement (on ne leur demande pas de faire un travaille d'étude à proprement parler ou ils chercheraient à incarner au mieux le personnage)
    \item Phase d'introduction:  Les participants sont accueillis et informés des détails de l'expérience à laquelle ils vont prendre part. On leur explique que leur objectif, dans chaque environnement virtuel qu’ils exploreront, sera d’utiliser la liberté d’action qui leur est offerte, s’ils en ont, pour créer un cadre dans lequel ils se sentiront le plus immergés dans le personnage qu’ils incarnent, un environnement qui leur permet de jouer et de donner du sens au texte qu'ils jouent. 
    \item Phase d'immersion: chaque participant va alors faire une suite de 6 immersions différentes au cours des 2 premiers jours de l'étude
    \begin{itemize}
    \item Phase d'adaptation: Le participant peut alors mettre le casque. Il sera d'abord en immersion dans une salle vide et sobre qui lui permettra de faire une transition fluide avec le monde virtuel et qui lui permettra de s'approprier l'espace et les différentes commandes (la téléportation et les menus de modifications de l'environnement)
    \item Phase d'exploration et de création de l'environnement: 
    \item Phase de jeu en réalitélité virtuelle
    \end{itemize}

\end{itemize}


