\section{Définition de la structure ITS et choix de la méthodologie d'apprentissage implicite}

\subsection{ITS}
Afin de donner un cadre plus clair à notre sujet, nous proposons de passer par la structuration de l'\textit{Intelligent Tuition System (ITS)}.
L'ITS est structuré en quatre modèles \cite{MASCARET.Buche+} : 

\begin{enumerate}
    \item modèle du domaine : représente la connaissance de l'expert à transmettre
    \item modèle apprenant : représente l'état de la connaissance de l'apprenant à tout instant
    \item modèle pédagogique :  représente les choix pédagogiques faits en fonction du comportement de l'apprenant 
    \item modèle d'interface : représente l'échangement d'information entre l'utilisateur et le système 
\end{enumerate}

Quelque soit le nombre de phases d'apprentissage du système que l'on choisit, le modèle du domaine reste le même, car la connaissance que l'on vise à transmettre à l'apprenant ne doit pas changer. 
En revanche, le modèle apprenant, lui, peut voir des différences d'implémentation ou d'approche, car la façon d'évaluer ou de modéliser la connaissance peut être sujette à des modifications. 
Les modèles pédagogiques et d'interface sont eux propres à chaque phase et approche retenue, et font partie des paramètres sur lesquels nous allons pouvoir jouer au cours des expérimentations. \\

Cette strucure permettra également de tester et comparer différentes interfaces et choix pédagogiques, que nous allons détailler dans la suite. 


\subsection{Apprentissage implicite}

Comme nous l'avons vu dans la partie \ref{definition_geste_expert}, le geste expert comporte à la fois une partie consciente et inconsciente.
Il semble donc logique de choisir des méthodes pédagogiques ciblant spécifiquement la mémorisation consciente et inconsciente, d'où pour la seconde l'attrait des méthodes d'apprentissage implicite. \\

Par définition, l'apprentissage implicite s'oppose à l'apprentissage explicite. 
L'implicite vise l'inconscient, sans un but apparent d'apprentissage clair et sans moyen d'auto évaluation, ce qui ne permet pas à tout instant de se situer dans son état d'apprentissage. 
Au contraire, l'explicite fournit un but, des moyens d'auto évaluation et l'état de connaissance de l'apprenant lui est connu, ce qui lui permet de choisir des stratégies et objectifs pédagogiques.
Ces deux modes d'apprentissage stimulent par ailleurs des aires du cerveau distinctes. \cite{ImplicitMotorLearning.Fang+}
On peut également voir l'apprentissage implicite comme le fait de retirer tous les éléments verbaux et analytiques du processus d'apprentissage, afin d'empêcher l'apprenant de stimuler sa mémoire explicite \cite{SoYouLearnImplicitly.Poolton+}
(ce qui peut également être fait en occupant artificiellement cette mémoire par quelque chose d'inutile pour la tâche en question, par exemple). \\

\cite{SoYouLearnImplicitly.Poolton+} cite trois grandes catégories de méthodes pour l'apprentissage implicite : 
\begin{itemize}
    \item apprentissage sans erreur : commencer l'apprentissage par des choses si simples que l'erreur y est improbable, pour ne pas ensuite pouvoir conscientiser une mauvaise manière de faire le geste
    \item focus externe de l'attention : ramener le focus de l'apprenant non pas sur son geste, mais sur la conséquence de son geste
    \item apprentissage par analogie : proposer une analogie du mouvement (par exemple sonification, mouvement similaire dans un autre domaine) pour exercer de façon décorellée de son domaine le mouvement \\
\end{itemize} 

L'apprentissage implicite présente également des avantages pédagogiques supplémentaire, car cette méthode propose une méthode de pédagogie positive engendrant un cycle vertueux d'apprentissage. 
En effet, cette plus grande autonomie de l'apprenant couplée à un focus externe de son attention offrent une plus grande motivation, une plus grande implication dans l'apprentissage, et permettent d'obtenir de meilleurs résultats en termes de précision et de rétention. \cite{OPTIMAL.Wulf+} 

