\paragraph{}
Dans certaines applications de réalité virtuelle, l'utilisation d'images et de vidéos panoramiques
 à 360° peut s’avérer pertinente. En effet, pour certains types d’environnements, une image statique 
 peut suffire à transmettre une ambiance ou une intention visuelle, sans nécessiter la construction 
 complète d’un espace 3D, souvent plus coûteux en termes de génération et de calcul. Ces dernières 
 années, la génération d’images de haute qualité a connu des avancées significatives 
 \cite{li2025comprehensive}, avec l’essor des modèles de diffusion qui tendent à supplanter les GANs (generative
 adversial networks). Dans un contexte immersif, la génération d’images par IA peut être 
 exploitée de différentes manières :

\begin{itemize}
\item \textbf{Création d’éléments 2D intégrés à la scène} : tels que des tableaux, écrans d’affichage, 
panneaux de signalisation ou affiches. Ces éléments peuvent enrichir visuellement l’environnement sans 
recourir à des objets 3D complexes.

\item \textbf{Génération de textures et de matériaux} : la création dynamique de textures permet de
modifier l’apparence d’un environnement (éclairage, température de couleur, ambiance générale) sans
altérer sa géométrie \cite{CHEN2024112113}.

\item \textbf{Création de décors de fond panoramiques} : la génération d’images 360° (skyboxes) à partir
 de descriptions textuelles \cite{chen2023text2lightzeroshottextdrivenhdr} permet de simuler des 
 paysages lointains ou des environnements secondaires. Bien que ces décors soient moins immersifs que 
 des scènes entièrement modélisées en 3D, ils restent efficaces pour compléter un espace visuel. Par 
 exemple, dans l’une des scènes de notre étude pilote, des skyboxes générées à l’aide de SkyboxAI 
 représentaient l’extérieur visible à travers les fenêtres d’un amphithéâtre virtuel.

\end{itemize}
