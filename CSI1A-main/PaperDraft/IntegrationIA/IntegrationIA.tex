\section{Intégration de l'IA}
À la suite de l’analyse des résultats et des retours issus de notre étude pilote, une prochaine étape 
essentielle consiste à intégrer des outils d’intelligence artificielle au sein de notre environnement
de répétition théâtrale en réalité virtuelle, et à interroger leur pertinence dans ce contexte spécifique.
L’IA offre en effet la possibilité de générer dynamiquement des décors virtuels en fonction du texte, 
de l’émotion recherchée ou de l’univers esthétique souhaité par le metteur en scène, et de les faire 
évoluer en temps réel selon les besoins du travail scénique. La nature des éléments générés, tout comme
les modalités d’interaction entre l’utilisateur et l’IA au cours des différentes étapes du processus de 
création, peuvent varier considérablement. Cette section se propose d’explorer ces possibilités et d’en 
analyser les enjeux.

\subsection{Nature des objets générés}
\subsubsection{Génération d'images et de vidéos 360 degrés}
\paragraph{}

Pour certaines utilisations en VR, les images et les vidéos 360 peuvent avoir leur interet. 
\subsubsection{Génération d'objets}
\paragraph{}
La génération d'objets 3D est une des utilisations les plus répandues de l'IA générative pour utilisation
en réalité virtuelle, que ce soit pour les développeurs mais aussi pour les utilisateurs. 
Afin de peupler un environnement existent, l'utilisateur pourrait avoir recours a de la génération d'objets 3D. 
En prenant l'exemple de notre étude pilote, le comédien pourrait avoir l'opportunité de prompter une IA pour
rajouter des éléments qui ne seraient pas présents dans le catalogue proposé. Il existe plusieurs natures d'objets 
3D et plusieurs facons de les générer:

\begin{itemize}
    \item Recherche d'objets présents dans une base de données existente: un des enjeux majeurs de la VR 
    reste la rapidité et l'économie de la puissance de calcul. Il est donc possible de réduire le temps de génération
    ainsi que la puissance de calcul nécéssaire grace a des algorithmes qui vont comprendre la requete de l'utilisateur
    et la comparer aux objets deja modélisés. C'est ce qu'entreprend un des blocs de l'architecture du LLMR, Large Language Model for Mixed 
    Reality (\cite{delatorre2023llmr}), en allant chercher dans la base de donnée SketchFab pour voir s'il existe un 
    modèle existent de l'objet demandé. 
    \item Génération d'objets Text to 3D: cette technique englobe tous les processus différents qui permettent 
    la génération automatiques d'objets 3D a partir d'une description textuelle. Bien qu'il existe un grand nombre
    de techniques différentes, le pipeline classique commence par l'interprétation sémantique du texte, suivie de la 
    synthèse d'une représentation volumétrique intermediaire (nuages de points, champ de densité implicite, image de profondeur), 
    avant d'etre converti vers un maillage 3D permettant le rendering. Parmis les techniques les plus récentes, il y a DreamFusion
    (\cite{poole2022dreamfusion}), qui utilise un modèle de diffusion text to image deja entrainé afin d'optimiser des modèles 3D initialisés aléatoirement (sous forme de Neural Radiance Fields) sans
    avoir recours a une base de données annotée de données 3D. OpenAI aussi travaille sur un modèle appelé Shap-E (\cite{jun2023shap}), qui génère des représentations
    implicites qui peuvent ensuite etre convertie en mesh ou en neural radiance field. 
    \item Génération d'objets animés et interactifs
\end{itemize}

dans l'etude pilote, les comédiens ont vocalisé l'envie/besoin d'avoir des objets 
animés/interactifs 
\subsubsection{Génération d'environnements 3D}
\paragraph{}
On peut aussi générer des environnements entiers 



\subsection{Nature de l'intervention (facon de communiquer avec l'IA)}
\subsubsection{Aucune intervention de l'utilisateur}
\subsubsection{Prompting textuel}
\subsubsection{Prompting vocal}
\subsubsection{Prompting gestuel}
\subsection{Niveau de l'intervention (personnalité de l'IA)}
\subsubsection{Prompted or independent AI}
\subsubsection{Incremental or instantaneous AI}
\subsubsection{Pleasing or provoking AI}