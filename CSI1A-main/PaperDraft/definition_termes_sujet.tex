\section{Définition du geste expert} \label{definition_geste_expert}

Le sujet, \textit{Interactions multimodales pour la transmission du savoir-faire artisanal en réalité mixte}, nous interroge tout d'abord sur la définition du savoir-faire, artisanal en l'occurence.
Pour cela, nous avons choisi de nous baser sur le concept du geste expert, le skill, dans le contexte de l'étude des mouvements humains. \\

Les caractéristiques fondamentales du geste expert sont les suivantes \cite{Reflecting_skill.Yadav+} : 
\begin{enumerate}
    \item sélection et exécution optimales du mouvement (A)
    \item grande précision du mouvement et de sa vitesse d'exécution (B)
    \item erreur et variabilité du mouvement réduites (B)
\end{enumerate}

où les indicateurs (A) et (B) renvoient à :
\begin{itemize}
\item (A) : une stratégie consciente de contrôle moteur, utilisé au début de l'apprentissage du geste expert en question
\item (B) : un processus inconscient qui permet de dépasser les contraintes de la tâche, résultant dans une performance bien au-delà du niveau de base
\end{itemize}

De cette définition découle l'idée que l'apprentissage du geste expert doit prendre en compte ces différentes caractéristiques, et en particulier ce clivage entre la partie consciente et inconsciente de la réalisation de ce geste expert. 
Nous pouvons déjà trouver dans la littérature des exemples de découpage en phase pour l'apprentissage du geste expert. 
Par exemple \cite{Coembodiment.Kodama+} propose les 3 phases d'apprentissage suivantes : 
\begin{enumerate}
    \item cognitive : comprendre le bon mouvement
    \item associative : reconnaître et pratiquer le bon mouvement
    \item autonomie : réaliser le geste de façon inconsciente
\end{enumerate}

Nous proposons donc, pour la suite, un découpage similaire en trois phases pour l'apprentissage, pour lequel nous avons donné les noms suivants :
\begin{enumerate}
    \item compréhension (a)
    \item préhension (b)
    \item autonomie (b)
\end{enumerate}

où les indicateurs (a) et (b) renvoient à:
\begin{itemize}
    \item (a) : un apprentissage explicite visant le conscient
    \item (b) : un apprentissage implicite visant l'inconscient
\end{itemize} 

Ce découpage nous permet de mieux tirer parti des particularités conscientes et inconscientes du geste expert. 
L'idée serait de mettre en place ce découpage au sein d'un système pédagogique, pour voir si découper l'apprentissage selon ces phases est en effet plus efficace et adapté pour l'apprentissage et la rétention qu'un système global.
Dans un deuxième temps, il faudrait ensuite essayer de proposer les meilleurs systèmes pour chacune des phases identifiées. 