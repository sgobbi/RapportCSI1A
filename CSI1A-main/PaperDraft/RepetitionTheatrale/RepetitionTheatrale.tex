\section{Répétition Theatrale}

Afin de comprendre les enjeux de notre projet, nous avons commencé par étudier les différentes techniques de répétitions utilisées 
dans un contexte de quete de sens du texte et d'ancrage dans le personnage a incarner. 
Dans le domaine de la répétition théâtrale classique, deux grandes écoles de pensée coexistent quant à la genèse du sens dans le 
jeu de l’acteur. La première école, incarnée notamment par la méthode de Stanislavski et les différents acteurs et metteurs en scène qui ont suivi dans
ses pas,
postule que le sens naît d’un travail intérieur profond : l’acteur explore ses émotions, ses souvenirs, et sa psychologie pour 
construire une vérité intérieure qui sera ensuite exprimée sur scène. Ainsi, le sens est 
d’abord trouvé à l’intérieur avant de pouvoir être extérieurement manifesté. À l’inverse, la seconde école considère que le processus 
débute par l’expression extérieure, en particulier par le mouvement du corps et la gestuelle.
Selon cette approche, le travail corporel précède et provoque l’émergence du sens intérieur, faisant de l’action physique le 
point de départ de la création dramatique. Ces deux perspectives offrent des voies complémentaires
pour comprendre comment le jeu théâtral s’élabore et donne vie au personnage.


\subsection{Recherche Exterieure: Travail du corps}
Plusieurs théoriciens du théâtre et de la répétition théatrale ont focalisé leur travail sur le corps et la recherche de sens a travers le
mouvement de celui-ci.

\begin{itemize}
    \item Meyerhold et la biomécanique
    \\
    Tout au long de sa vie, Meyerhold vient replacer le corps au coeur du jeu en le considérant comme le premier vecteur d'émotion et de pensée. Le geste, 
    la posture et le respiration précèdent l'émotion interieure et vont l'engendrer chez le comédien \cite{PiconVallin}. Il élabore alors la biomécanique,
    une série d’exercices physiques codifiés, inspirés à la fois du sport, de la danse, du cirque et du théâtre oriental, dont le but
    est d'atteindre la perfection expressive selon les critères suivant: le rythme et le synthèse gestuelleé. 
    Ces exercices, précis et répétitifs, visent à affiner la coordination, l’équilibre, la rapidité d’exécution et la conscience spatiale de l’acteur.
    Pour Meyerhold, ce travail technique et stylisé permet à l’acteur de développer un contrôle total de son corps, condition essentielle pour incarner 
    un rôle avec justesse, efficacité et expressivité.
    Dans cette recherche, il va alors développer des exercices précis parmi lesquels figurent des séquences de mouvements codifiés, comme le tir à l’arc,
    le lancer, la gifle ou le transport de charge \cite{Baldwin}. Ces études sont construites selon une structure précise en quatre temps : Otkaz (la préparation du mouvement,
    un recul nécessaire pour créer l’élan), Posil (l’action ou l'impulsion principale), Stoïka (la fixation ou l’arrêt du mouvement, qui marque sa fin et en souligne la forme), 
    et parfois Tormoz (le ralentissement ou retour à l’équilibre). Chaque phase est essentielle pour rendre le mouvement lisible, expressif et techniquement maîtrisé.
    Ces exercices permettent à l’acteur de prendre conscience de chaque intention gestuelle, de mieux contrôler son corps, et d’apprendre à générer une énergie dynamique au service du jeu scénique.

    \item Grotowski et le théâtre pauvre
    \\ 
    Le travail de Jerzy Grotowski vient s'inscrire dans la continuité de celui de Meyerhold dans sa quête d’un théâtre essentiel \cite{grotowski}. 
    Il développe une approche radicale où le sens du personnage ne naît pas d’un texte ou d’une psychologie préétablie, mais d’un travail corporel 
    profond et organique. Pour lui, l’acteur devient un « athlète de l’émotion », qui accède à l’authenticité du jeu par une ascèse physique et vocale rigoureuse.
    Contrairemement à Meyerhold qui cherchait l'eeficacité d'un corps entrainé, Grotowski cherche la vérité interieure à travers un 
    dépouillement de tout ce qui est superflu --- les automatismes, les protections sociales --- pour faire surgir un corps traversé par une vérité interieure ou chaque 
    geste contient une necessité et intention profonde. 

    \item Jacques Lecoq et le masque neutre
    \\
    A cette époque, un mouvement de travail autour du corps nait en France aussi avec Jacques Lecoq et son 
    travail sur le masque neutre qui s'inscrit dans une tradition de théâtre du mouvement influencée par Copeau, Artaud,
    et le théâtre oriental, notamment le Nô japonais. Issu du monde de l’éducation physique, Lecoq développe une pédagogie
    basée sur une approche corporelle rigoureuse, dans laquelle le masque neutre joue un rôle fondamental \cite{kemp2012embodied}. Hérité du 
    "masque noble" découvert auprès de Jean Dasté, ce masque impose une neutralité d’expression qui prive l’acteur de son 
    visage comme vecteur d’émotion, et l’oblige à chercher l’essence du mouvement, à exprimer avec justesse et clarté par le rythme, 
    la respiration et l’attitude corporelle. Ce dépouillement crée un point zéro du jeu, permettant de révéler les lois fondamentales 
    du théâtre à travers le mouvement. Comme il le dit lui-même : « C’est en enseignant que j’ai découvert que le corps sait des 
    choses que la tête ne sait pas encore » \cite{lecoq2001moving}. 
\end{itemize}

\subsection{Recherche Interieure: Le travail du psychisme} 
Si certains pédagogues ont centré leur travail sur le mouvement et l’expressivité physique 
comme point de départ du jeu, d'autres ont exploré une voie plus introspective. Dans cette 
perspective, le sens théâtral naît d’un travail intérieur, où l’acteur puise dans son vécu 
émotionnel et psychologique pour construire un personnage authentique.
\begin{itemize}
    \item Le système de Stanislavski
    \\ 
    C'est le cas de Konstantin Stanislavski, professeur d'art dramatique russe et théoricien de la 
    répétition théâtrale, qui tout au long de sa vie explore et développe son système qui
    vise à instaurer un jeu réaliste et vivant sur scène, en s’éloignant de l’artifice théâtral et 
    des performances mécaniques. Pour cela, il élabore une méthode dans laquelle l’acteur doit s’immerger
    sincèrement dans la psychologie de son personnage en mobilisant ses propres expériences de vie. L’un 
    des outils clés de ce système est la mémoire affective, qui consiste à faire appel à des souvenirs
    personnels chargés d’émotion afin de nourrir le jeu de manière authentique. En ravivant ces émotions vécues, 
    l’acteur parvient à incarner son rôle avec véracité et intensité, rendant ses réactions sur scène plus crédibles
    aux yeux du spectateur \cite{Stanislavski}.
    Dans ce contexte de recherche de mémoire affective, il développe des exercices précis basé sur les sensations et les 
    souvenirs de celles-ci. Par exemple, il invite l’acteur à revivre intérieurement une situation marquante de son passé, 
    puis à transposer l’émotion ressentie dans une scène donnée. Il propose également des exercices de relaxation sensorielle,
    où l’acteur s’isole, ferme les yeux et se concentre sur les détails sensoriels d’un souvenir, afin de réactiver les 
    émotions associées et d’en nourrir son jeu.  
    \\
    Dans les dernières années de sa vie, Stanislavski remet en question certains aspects de sa première méthode, notamment
    la prééminence de la mémoire affective comme unique point d’entrée dans le jeu. Il constate que cette recherche intérieure, 
    bien qu’essentielle, peut parfois enfermer l’acteur dans un travail trop cérébral ou inhiber l’élan du jeu. Il développe alors 
    ce qu’il appellera la méthode des actions physiques, une approche dans laquelle le corps en mouvement et l’action concrète sur 
    scène deviennent un point de départ du travail émotionnel. Ce basculement marque une évolution majeure dans sa pensée : 
    ce n’est plus l’émotion qui génère l’action, mais l’action qui révèle l’émotion. Maria Knebel, l'une de ses élèves, poursuivra 
    cette voie en formulant la méthode de l’analyse-action \cite{knebel2006analyseaction}, qui propose d’explorer le texte par la 
    pratique, en se concentrant sur les objectifs et les actions du personnage plutôt que sur une analyse psychologique théorique.
    \item La methode de L'Actors Studio
    \\
    Lee,Strasberg, fondateur de la légendaire méthode de l’Actors Studio, s’inscrit clairement dans la lignée de Stanislavski,
    mais pousse l’exploration vers une discipline plus affirmée et sensorielle. Comme Stanislavski, il met l’accent sur la mémoire
    affective et la concentration sensorielle, mais il structure ces pratiques en un training quotidien rigoureux, incluant des exercices de 
    relaxation, des "foundation sense memory exercices", "advanced sense memory exercices", et des "character development exercices",
    transcrits par Lola Cohen \cite{cohen2016method}. 
    Voici quelques exemples des exercices présents dans la méthode:
    \begin{itemize}
        \item Exercice de relaxation: pendant cet exercice, le comédien va s'affaler dans une chaise, faire un scan de tout son corps 
        en relachant endroit par endroit, avec pour but de trouver la tension dans son corps et s'en défaire completement en arretant 
        tous mouvements parasites
        \item Sunshine Exercice (foundation sense exercice): cet exercice consiste à faire un premier travail au soleil en le sentant sur 
        tout son corps, puis un deuxième travail à l'ombre en essayant de ressentir ces mêmes sensations.  
        \item Place exercice (advanced sense memory exercice): cet exercice est un exercice d'imagination  lors duquel le comédien se
        place mentalement dans un lieu connu (sa maison d'enfance par exemple) et se balade dans l'espace qu'il occupe comme s'il s'agissait 
        du lieu de son imaginaire, en essayant d'intéragir avec tous les potentiels objets et de le décrire au mieux.
        \item  Improvisation exercice (character exercices): lors de ces exercices d'improvisation autour du personnage, le comédien va 
        jouer des scènes qui n'existent pas dans le texte de la pièce, des évènements se passant avant ou après l'action de la pièce, afin 
        de construire de la familiarité avec son personnage. 
    \end{itemize}
    Il transforme ainsi l’art intérieur en un processus empirique et méthodique, 
    où l’émotion naît de la re-création sensorielle — un prolongement direct mais affiné de la première partie du système de Stanislavski.
    
\end{itemize}

\vspace{\baselineskip}

Les multiples approches que nous avons étudiées – de Meyerhold à Strasberg – révèlent la richesse et la diversité des méthodes permettant 
à l’acteur d’accéder au sens dramatique. À présent, il s’agit d’interroger comment ces processus, historiquement ancrés dans une pratique
physique et sensorielle du plateau, peuvent être traduits, adaptés ou réinventés dans un environnement numérique. Nous allons donc explorer 
dans la partie suivante les possibilités qu’offre la réalité virtuelle pour réinventer la répétition théâtrale et prolonger cette quête de 
sens dans un espace immersif.

  




